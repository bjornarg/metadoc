\newpage
\section{Overview}
\label{sec:overview}
Usage of the MetaDoc client is done mainly through the use of \texttt{mapi.py}.
\texttt{mapi.py} takes care of sending and retrieving data to and from the
server, caching any data that could not be sent, and validating \gls{xml} data
recieved.

\subsection{Configuration}
\label{sec:metadoc_conf}
The MetaDoc client uses a configuration file \texttt{metadoc.conf}. This file
\textit{must} be placed in the same folder as the client itself. Listing
\ref{lst:config} provides an example of a basic configuration file.

The configuration is in INI format and defines a section named
\texttt{MetaDoc}.

\paragraph{Required parameters:}

\begin{description}
    \item[host] \textbf{baseurl} for the MetaDoc Server that the client should
        communicate with.
    \item[key]  Absolute path to the clients X.509 certificate key file. This
        should be the private key for \textbf{cert}, and is used to encrypt
        data passed to the server.
    \item[cert] Absolute path to the clients X.509 certificate file. This is used
        to authenticate the client with the server. See section
        \ref{sec:authentication} for more information.
    \item[site\_name]   The name of the site the client is running on. 
\end{description}

\paragraph{Optional parameters:}

\begin{description}
    \item[trailing\_slash]  Whether the client should append a trailing slash
        at the end of URLs used to connect to the server. At the moment, this
        should be set to \texttt{True}.
    \item[valid]    Initially set to \texttt{False} when \texttt{mapi.py}
        creates a sample configuration file. This is set to avoid running the
        client without being properly configured. If this value is present, it
        \textit{must} be set to \texttt{True} or \texttt{yes}.
\end{description}

\texttt{mapi.py} will create a initial configuration file if one is missing
when it starts. This can be used as the basis for a configuration file. 

\scriptcode[lst:config]{Example of a basic configuration file}{examples/configuration/metadoc.conf}{}

\subsection{Handles}
\label{sec:handles}
A handle is an option passed to the client when running the script.

\texttt{mapi.py} takes handles that tell the script what information to send
or retrieve to or from the server. All handles can be mixed together
\textit{except} for handles that override each other. Handles that overrides
others are explicitly stated below.

\texttt{mapi.py} takes the following handles:

\begin{description}
    \item[-h, --help]   Displays a short help message explaning the handles
    that may be passed to \texttt{mapi.py}. Overrides any other handles.
    \item[-v, --verbose]    Verbose mode. Prints information about progress and
    information sent and recieved between client and server. 
    \item[-q, --quiet]  Quiet mode. Prints nothing unless the program fails.
    Overrides \textbf{-v}, \textbf{--verbose}.
    \item[-l \textless log level\textgreater, --log-level=\textless log
    level\textgreater] Sets the log level for the program. See section 
    \ref{sec:logging} for more information about what is logged at different 
    levels.
    \item[-n, --no-cache]   Prevents the client from sending any cached data.
    If any errors occour when the client runs with this handle, data from this
    run will \textit{not} be cached. For more information about caching, see
    section \ref{sec:caching}.
    \item[-e]   Sends event data from client to server.
    \item[-c]   Sends configuration data from client to server.
    \item[-s]   Sends software data from client to server.
    \item[-u]   Retrieves user data from the server.
    \item[-a]   Retrieves allocation data from server.
    \item[-p]   Retrieves project data from server.
    \item[--dry-run]    Does a dry run, not sending any data to server. Does
        not retrieve cached data, and does not save any data to cache.
        Should be run with verbose to see data that would be sent.
    \item[--all]    Sends and retrieves all possible data. Equal to -ecsuap.
    \item[--send-all]   Sends all possible data. Equal to -ecs.
    \item[--fetch-all]  Retrieves all possible data. Equal to -uap.
\end{description}

\subsection{Logging}
\label{sec:logging}
The client logs data to \texttt{/var/log/mapi/}. The folder must be readable
and writable for the user that runs the.  The client creates a new log file
each day it runs, with the naming schema:
\texttt{metadoc.client.YYYY-mm-dd.log}. 

The client has five different logging levels. The list below gives an overview
of what is logged at the different levels. The higher items in the list contain
everything below as well, so that with a log level set to \textbf{error} will
also contain \textbf{critical} logging.

The log level defaults \textbf{warning}. 

\begin{description}
    \item[debug]    Debugging information, used for development and error
    checking.
    \item[info] Information about what is happening during execution, such as
    items sent or recieved to/from the server.
    \item[warning]  Warnings occouring during execution, mainly problems that
    will not cause a failure but that should be addressed.
    \item[error]    Errors that cause partial failure of the execution, such as
    being unable to connect to the server.
    \item[critical] Critical failures that causes the execution to halt, or
    errors in the program code itself.
\end{description}

\subsection{Caching}
\label{sec:caching}
The client caches data to \texttt{/var/cache/mapi/}. Files are named after the
data type that is cached in each file. The user running the client must have
read and write access to this folder. 

The client will cache any information that is not accepted by the server, 
\textit{unless} the server returns a reciept for the information that marks the 
information as invalid or malformed in some way, such that the information will 
not be accepted if resent at a later date. See section \ref{sec:errors} for
more information about errors.

Data the client sends may be marked so that it will not resend any cached data
when the client is run with the same handle. This is mainly for use for full
updates, such as software and configuration, where any cached data would be
outdated or duplicates if sent together with a new run.

If the \textbf{-n} or \textbf{--no-cache} handles are passed, the script will
ignore any cached data completely and run as if it didn't exist. The client
will also \textit{not} cache any data on this run. The cached data will then be
processed on the next run where \textbf{-n} or \textbf{--no-cache} is not
passed. 
