\newpage
\section{XML document}
\label{sec:xmldoc}
The \gls{xml} document should follow the form described in the MetaDoc
\gls{dtd} \cite{metadoc_dtd}. Below certain conventions used in the \gls{xml}
build is explained. Any alterations to the \gls{dtd} should follow these
conventions in order for the client and server to continue functioning
normally. 

\subsection{Document build}
Any type of information sent should only create one direct child of the root
element, \texttt{<MetaDoc>}. This means that when lists of information is sent,
the list elements should be placed within a container element, and \textit{not}
directly in the root element. The container element should have an self
explanatory name about the information passed. 

An example is that \texttt{<user\_entry>} elements are placed within a
\texttt{<users>} element. Here \texttt{<users>} is considered the container
element, and there should only be one of them in each MetaDoc \gls{xml}
document.  \texttt{<users>} may contain any number of \texttt{<user\_entry>}
elements. 

\subsection{Dates}
All dates in the document should be on the form specified by RFC3339
\cite{rfc3339}. The \texttt{utils} module provides a function
\texttt{date\_to\_rfc3339} that takes a \texttt{datetime.datetime} object and
returns a string on RFC3339 form. It also provides a function
\texttt{rfc3330\_to\_date} which will return a \texttt{datetime.datetime}
object from a proper RFC3339 string, or \textbf{False} if the string is not a
correct RFC3339 date.

\subsection{Special attributes}
The \textbf{id} attribute of elements have a special function in MetaDoc. This
attribute is used to identify the object when receiving receipts from the
server whether elements have been added. The attribute is \textit{not} saved in
caching to avoid duplicate \textbf{id}s when resending cached data together
with new data. If you want to give elements a special identifier that should be
saved, it must be called something other than \textbf{id}.
