\newpage
\section{MetaDoc Server API}
\label{sec:server_api}

The MetaDoc server implements a REST-like API, however, there are certain
differences from REST noted in section \ref{sec:diff_from_rest}.

When the client performs a GET request on an availible URL, the server should 
return an XML document, or a HTTP status code refering to an error. 
The XML document should follow the MetaDoc DTD \cite{metadoc_dtd}. Each URL
only returns data from the requested data type. This means that a request to
\textbf{baseurl/allocations/} will return an MetaDoc XML document containing
only an \texttt{<allocations>} directly on the \texttt{<MetaDoc>} root, with
\texttt{<all\_entry>} tags as children of \texttt{<allocations>}. The client
should disregard any information outside \texttt{<allocations>} when connecting
to \textbf{baseurl/allocations/}. 

In order to send data to the server, the client performs a POST request, with
the POST data variable \texttt{metadoc} containing a MetaDoc XML document. The
server will only accept data from the data type specified in the URL, and will
disregard any other information. This means that a POST to
\textbf{baseurl/events/} should be a MetaDoc XML document containing a 
\texttt{<events>} tag directly on the \texttt{<MetaDoc>} root, with any number
of \texttt{<resourceUp>} and \texttt{<resourceDown>} tags as children of
\texttt{<events>}. 

When this data is sent to the server, the server should return a MetaDoc XML
document containing a \texttt{<receipts>} tag, with a \texttt{<r\_entry>} tag
for each element recieved that has an \textbf{id}-attribute. This allows for a
very fine grained error reporting.

\subsection{Available URLs}

\begin{description}
    \item[/allocations/] Retrieves a list of allocations/quotas relevant
        to the client
    \item[/users/] Retrieves a list of users for the client
    \item[/projects/] Retrieves a list of projects relevant to the
        client
    \item[/config/] Sends system configuration to server
    \item[/events/] Sends site events to the server
    \item[/software/] Sends system software to server
\end{description}

\subsection{Authentication}
\label{sec:authentication}
The site uses X.509 certificates to authenticate the client. In order for the
site to be authenticated properly, the server must be aware of the client's
certificate prior to the request, and the correct owner of the certificate must
be saved on the server. This \textit{must} be the same as the value for
\texttt{site\_name} set in the MetaDoc configuration (see section
\ref{sec:metadoc_conf}).

\subsection{Differences from REST}
\label{sec:diff_from_rest}

There are certain differences in the API compared to the REST specification. The 
MetaDoc Server API makes use of HTTP POST where HTTP PUT should be used in 
accordance with REST. This is due to limitations in standard Python libraries.

Because the MetaDoc Server does not give the client access to delete or replace
data on the server, this does not create problems for the Server API. % FIXME

\subsection{Server HTTP responses}

The server makes use of HTTP status codes to identify what error has occoured
if the server is unable or unwilling to process the request from the server.
Table \ref{tbl:http_status_codes} contains a list of status codes the server
returns, and why these status codes occour. 

\begin{table}[h!]
    \centering
    \caption{List of HTTP status codes used by the MetaDoc Server}
    \begin{tabular}{|l|l|p{7cm}|}
        \hline
        \textbf{Code} & \textbf{Official name} & \textbf{MAPI-Reason} \\
        \hline
        403 & 403 Forbidden & The client certificate was not recognized as an
        authorative source for the site given in the XML document. \\
        \hline
        500 & 500 Server Error & The server failed to process the request. This
        does not include errors that return a receipt. \\
        \hline
    \end{tabular}
    \label{tbl:http_status_codes}
\end{table}
