\documentclass[titlepage, a4paper,10pt]{article}
\usepackage[english]{babel}
\usepackage[utf8x]{inputenc}
\usepackage{listings}
\usepackage{graphicx}
\usepackage[dvips]{hyperref}
\usepackage{glossaries}
\hypersetup{bookmarks = true}
\hypersetup{pdftitle = MetaDoc Quick start guide}
\hypersetup{pdfauthor = Bjørnar Grip Fjær}
\hypersetup{pdfsubject = MetaDoc Quick start guide}
\hypersetup{colorlinks = true}
\hypersetup{urlcolor = black}
\hypersetup{linkcolor = black}
\hypersetup{citecolor = black}
\hypersetup{filecolor = black}
\addtolength{\oddsidemargin}{0.85cm}
\addtolength{\evensidemargin}{-1.5cm}
\addtolength{\textwidth}{0.75cm}
%\addtolength{\topmargin}{-1.0cm}
\addtolength{\textheight}{1cm}
\newcommand\rawcode[3]{{#1 \lstinputlisting[label=#2]{#3}}}
\newcommand\scriptcode[3]{
        \lstset{language=#3,
                emph={asmlinkage, \_\_user, ENTRY, foreach, \_\_init, \_\_exit},
                numbersep=15pt,
                numbers=left,
                numberstyle=\scriptsize,
                frame=tb,
                tabsize=8,
                commentstyle=\texttt,
                keywordstyle=\bfseries,
                emphstyle=\bfseries,
                linewidth=0.90\textwidth,
                showstringspaces=false
                frame=shadowbox,
                rulesepcolor=\color{blue}}
                \rawcode{\scriptsize}{#1}{#2}}
\newcommand\inlinecode[1]{{ \texttt{\small #1} }}

\newacronym{mapi}{MAPI}{MetaDoc API}

\title{MetaDoc\\Quick start guide}
\author{Bjørnar Grip Fjær}
\date{\today}


\begin{document}
\maketitle

\section{Introduction}
This guide is meant to provide a quick introduction to using MetaDoc and the
\gls{mapi}. For more complete information about MetaDoc, please refer to the
MetaDoc Documentation \cite{mdoc}.

\section{Retrieving the MetaDoc client}
Before you start you need a version of the MetaDoc client. You can download the
latest version from the MetaDoc GitHub repository \cite{downloads}.

Extract the client to the wanted destination. 

\section{Setting up folders}
The MetaDoc client uses two system folders, namely \texttt{/var/log/mapi/} and
\texttt{/var/cache/mapi/} for log files and cache files, respectively. Make
sure to create these folders, and make sure the user running the MetaDoc client
has access to \textbf{read} and \textbf{write} to these folders. 

\section{Configuring the MetaDoc client}
The MetaDoc client uses a configuration file called \texttt{metadoc.conf} in
the same folder as the client itself. You can create a sample configuration
file by running \texttt{mapi.py} without any parameters. 

The configuration file contains the following settings:

\begin{description}
    \item[host] URI of a server responding to the MetaDoc server API
        \cite{mdoc}.
    \item[key]  Location of file containing the private key for the client.
    \item[cert] Location of file containing the certificate for the client key.
    \item[site\_name]   Name of the site. This must correspond to the site name
        set on the server for the given key.
    \item[trailing\_slash]   Whether the server uses a trailing slash on URIs,
        should be set to True for now.
    \item[valid]    Defaults to False to avoid the script from running with
        default configuration. Should be set to True or removed once properly
        configured.
\end{description}

Please note that file locations \textbf{must} be given as an absolute path.

The certificate file referenced by the \textbf{cert} key in the configuration
must be available at the server and related to your client properly, otherwise
the server will refuse the connection.

\section{Using the information}
Once the actions detailed above have been performed, the MetaDoc client is
ready to send and receive information. However, it has to know what to do with
the information it receives, and how to gather information to send. 

\texttt{doc/examples/} contains examples of customized functions that produce a
shadow file from user data received, project user file from project data and
project quota file from received allocation data. It also contains an example
of a command line tool for adding events.

Please refer to the MetaDoc documentation \cite{mdoc} for more information on
customizing the MetaDoc client. 

\section{Setting up crontab}
The MetaDoc client can be set up as a crontab in order to synchronize sites
with the server. Below, some examples of using the client in crontab is shown.
The information passed between the client and server depends on the handles
passed on script execution. 

A couple of useful handles are \textbf{--send-all} and \textbf{--fetch-all},
which sends all data and retrieves all data, respectively. \textbf{--all} is a
combination of these.  See the MetaDoc documentation for a detailed explanation
of each handle \cite{mdoc}.

\begin{verbatim}
0  0  *  *  *  /path/to/metadoc/client/mapi.py -ecs -q -l error
\end{verbatim}
Sends event, configuration and software data to the server at midnight every
night. Makes sure the script executes quietly, passing no output to
\texttt{stdout}, only to \texttt{stderr} if the script execution fails. Only
logs error messages.

\begin{verbatim}
0  0  *  *  0  /path/to/metadoc/client/mapi.py -upa -q -l warning
\end{verbatim}
Fetches user, project and allocation data once a week. Makes sure the script
executes quietly and logs warning messages.

\section{Logging}
The MetaDoc client logs to \texttt{/var/log/mapi/}, log files follow the naming
schema \texttt{metadoc.client.YYYY-mm-dd.log}. The amount of information logged
depends on the log level passed to \textbf{-l} on script execution. The
possible log levels are \textbf{debug}, \textbf{info}, \textbf{warning},
\textbf{error} and \textbf{critical}. When \textbf{-l} is not passed, or passed
with a log level not available, the level is set to \textbf{warning}. See the
MetaDoc documentation for more information on logging \cite{mdoc}.

\newpage
\begin{thebibliography}{99}
    \bibitem{mdoc} Fjær, Bjørnar Grip, \textit{MetaDoc Documentation},
        \url{http://bjornar.me/metadoc/doc.pdf}
    \bibitem{downloads} \textit{MetaDoc Downloads},
        \url{http://github.com/bjornarg/metadoc/downloads}
\end{thebibliography}


\end{document}
