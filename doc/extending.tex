\newpage
\section{Extending MetaDoc}
\label{sec:extending}
MetaDoc is explicitly designed to allow for future enhancements in case more
information should be sent. In order to do so, an series of steps are required.
The list below details each of these steps, and each step is explained in more
detail in the following sections.

\begin{enumerate}
    \item
        Extend the MetaDoc \gls{dtd} with the definition of the data.
    \item
        Creating definition files on the MetaDoc Client.
    \item
        A \texttt{MetaInput} or \texttt{MetaOutput} sub class that should
        handle data, depending on whether data is recieved or sent to or from
        the server, respectively.
    \item
        Adding a handle to \texttt{mapi.py} that will activate the data type.
    \item
        Configuring the server to send or recieve the intended data.
    \item
        Updating the version of MetaDoc.
\end{enumerate}

Figure \ref{fig:connection_example} on page \pageref{fig:connection_example}
shows an example of how the data for projects is defined, and the connection
between classes that are used when data about projects is recieved from the
server.


\subsection{Altering DTD}
\textit{The \gls{xml} document follows certain conventions that should be
followed when extending the \gls{dtd}. These conventions are explained in more
detail in section \ref{sec:xmldoc}.}

Before you alter the \gls{dtd} you should know exactly what data should be
sent.  Create an \texttt{<!ELEMENT>} with a descriptive name of the data sent.
As an example, \texttt{<users>} is used for a list of users. 

Add any attributes necessary to describe the set of data. If the data is a list
of entries, such as a list of users, create an \texttt{<!ELEMENT>} as a
possible sub-element that contains the information about each entry. Any short
information about the entry should be placed in attributes of the entry. If
there is more information, such as information that could be several sentences
or lines, it should not be placed as an attribute. This information should be
placed inside the element itself. If there are several types of long
information for each entry, create a descriptive \texttt{<!ELEMENT>} for each
as a sub-element of each entry to contain the text. Otherwise the text may be
placed directly inside the entry element itself. 

As an example, the MetaDoc \gls{dtd} \cite{metadoc_dtd} defines
\texttt{project\_entry}, where \texttt{remarks} and \texttt{description} are
allowed sub-elements that contain any text. Meanwhile, \texttt{resourceUp} and
\texttt{resourceDown} places the text directly inside the tag itself, as not
other information is allowed inside these tags.

\textit{Make sure the updated \gls{dtd} is placed on both the client and
server.}

\subsection{Defining the data client side}
\label{sec:defclientmodel}
Add a package \cite{python_modules} to the client with the name of the main
element. Create a module \texttt{definition} inside this package.
\texttt{definition} should define the main element with a sub class of
\texttt{metaelement.MetaElement}.

Create a module \texttt{entries} inside the same package. This file should
contain definitions of each entry, and potential sub elements for each entry,
as a sub class of \texttt{metaelement.MetaElement}. 

Add the entry class(es) to \texttt{self.legal\_element\_types} of the \\ 
\texttt{metaelement.MetaElement} sub class defining the main element. 

\texttt{metaelement.MetaElement} sub classes may define a
\texttt{clean\_<attribute name>()} for each attribute on the element. This
method will recieve the attribute value, and should return the attribute value
after any potential cleaning is done on it. Please note that \textit{all}
attribute values \textit{must} be strings, so if any value set as an attribute
might be set as anything other than a string, the clean-function is the place
to convert it. If the attribute contains a value that is not allowed, a sub
class of \texttt{metaelement.IllegalAttributeError} should be raised that
defines the error that has occoured. 

The \texttt{metaelement.MetaElement} defines some methods that are commonly
used in cleaning methods, such as converting a \texttt{date} object into an
RFC3339 string, or checking for legal values of an attribute. See section
\ref{sec:useful_classes} for more information about these functions and how to
build these classes.

\subsection{Custom client handles}
If the data is to be sent from client to server, create a module
\texttt{custom/site<main element name>.py} that contains a sub class of
\texttt{custom.abstract.MetaOutput}. This class should define a method
\texttt{populate()} which gathers the information to be sent from the site and
appends an instance of a entry-class, as defined in section
\ref{sec:defclientmodel}, to \textbf{self.items} for each entry.

If the data is sent from server to client, a module called
\texttt{custom/update<main element name>.py} should be created that contains a
sub class of \\ \texttt{custom.abstract.MetaInput}. This class should define a
method \texttt{process()} that processes any recieved data in
\textbf{self.items}.  

See section \ref{sec:client_api} and \ref{sec:useful_classes} for more
information on these classes.

\subsection{Updating \texttt{mapi.py}}
A couple of changes are necessary inside \texttt{mapi.py} in order for the
script to be able to send or recieve the new data added. 

First, a new handle has to be created for the new data. The handle should
preferably be descriptive to what the data transferred is (e.g. \textbf{-u} for
user data), but it \textit{cannot} conflict with any of the existing handles.
The handle should be added to \texttt{mapi.py}'s docstring, the \textbf{optstr}
and potentially \textbf{optlist} variables inside the \texttt{main()} function,
and must register the element inside the \textbf{opts} loop. 

If the data is to be sent from the client, the definition class for the
container element (see section \ref{sec:defclientmodel}) should be added to the
\texttt{main()} function's \textbf{possible\_send\_elements} variable. It the
data is to be retrieved from the server, add the class to
\textbf{possible\_fetch\_elements} instead.

\subsection{Updating the MetaDoc Server}
The MetaDoc server must respond to the URL defined by the \textbf{url} variable
defined on the definition class of the data on the client (see section
\ref{sec:defclientmodel}). The URL must conform to the MetaDoc Server
\gls{api}. See section \ref{sec:server_api} for more information about the
MetaDoc Server \gls{api}.

\subsection{Versioning}
\label{sec:version}
MetaDoc passes a \textbf{version} attribute on it's root element,
\texttt{<MetaDoc>} when sending information between client and server. This
version is a string on the form ''\texttt{X}.\texttt{Y}.\texttt{Z}'', where
\texttt{X}, \texttt{Y} and \texttt{Z} are numbers. Changes made to each number 
indicate different levels of breakage. 

When extending the MetaDoc \gls{api}, you need to consider what version level
has to be changed. Section \ref{sec:version_levels} explains each of the
levels.

\textit{Any updates to the MetaDoc \gls{api} must update the version.} Failure
to do so may result in unreliable consequences.

\subsubsection{Version levels}
\label{sec:version_levels}
When \texttt{X} is changed, changes are made such that the current information
passed is changed in some way. This may be changes to the \gls{dtd} where any
of the currently passed information is affected (addition/removal of
attributes, changes to how attribute values are presented or should be parsed,
addition/removal of sub-elements). If the client or server encounters a
document with a different value of \texttt{X} in the version number, it should
\textit{not} accept the data, as it cannot be sure it will handle it correctly.

Changes to \texttt{Y} indicates a change that does \textit{not} change the
current behaviour in any way, but may be instances where new information might
be passed. When the client or server encounters a document with a different
value of \texttt{Y} it should log a warning, but otherwise proceed as normal.

\texttt{Z} is currently not used for anything, but is present for potential
usability in the future. Differences in \texttt{Z} should be logged as debug
information.

\subsubsection{Changing the version}
The version is set in the \texttt{version} module's \textbf{\_\_version\_\_}
variable. This is the variable that must be updated to update the version of
the MetaDoc Client.

Make sure the MetaDoc server is also updated to the same version.
