\newpage
\section{Errors}
\label{sec:errors}

The server returns a \texttt{<receipt>} containing an \texttt{<r\_entry>}
\textit{for each} element passed. The \texttt{<r\_entry>} has the required
attributes \textbf{id} and \textbf{code}, containing the ID of the element sent
by the client and the error code, respectively. It may also contain an
attribute \textbf{note} with a short note explaining the error if extra
information is available. The \texttt{<r\_entry>} tag might also contain text
with a longer message, if more information is needed about the error. 

An example of a error message would be an error code \texttt{2001} with the
note \texttt{Missing attribute "reason"}.

A list of possible return codes is given in table \ref{tbl:server_return_codes}
in appendix \ref{app:server_return_codes}. The difference between critical and
non-critical errors are that critical errors are problems with the element
itself, causing it to be refused by the server at any time. Non-critical errors
are errors where the element would potentially be accepted at a later date, but
not now.

\subsection{Document errors}

In the special case where there are problems with the document itself, such as
\gls{xml} errors or the document not passing \gls{dtd} verification, the
\texttt{<r\_entry>} \textbf{id} attribute will be set to \texttt{0} (zero),
referring to the document itself. 
